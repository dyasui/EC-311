\begin{question}[type=exam]{25}
  Jose's utility function for bundles of Rice, $R$, and Beans, $B$, is:
  $$ U(R,B) = R^{\vary{1/4}{3/4}} B^{\vary{3/4}{1/4}} $$

  \begin{enumerate}[label=\alph*)]

    \item (\points{5})
      Set up Jose's utility maximization problem and derive the optimality condition.
      Keep prices and income as variables for now.
    \vspace{8cm}

    \item (\points{5})
      Suppose that Jose only has \$8 to allocate between Rice and Beans
      and that the price of both Rice and Beans is equal to \$1.
      How many units of each good will Jose consume?

    \newpage

    \item (\points{5})
      Illustrate your answer to part (b) in the space provided below.
      Make sure to include the budget constraint, optimal bundle, and an indifference curve and label them appropriately.

      \begin{center}
      \begin{tikzpicture}
      \begin{axis}[
        width=0.8\textwidth,
        grid,
        xlabel = {Beans},
        ylabel = {Rice},
        xmin=0, xmax=8,
        ymin=0, ymax=8,
        xtick={1,2,...,8},
        ytick={1,2,...,8},
        grid style=dashed,
        ]
        
        \addplot[draw=none] coordinates {(1,1)};

      \end{axis}
      \end{tikzpicture}
      \end{center}

    \item (\points{5})
      Now suppose that the price of \vary{Beans}{Rice} goes from \$1 to \$3.
      What will Jose's new consumption of \vary{Beans}{Rice} become?
      Add the new budget constraint and indifference curve generated by the price change to your graph above.
    \vspace{3cm}

    \item (\points{5})
      How large is the size of the \textbf{income effect} relative to the \textbf{substitution effect} in part (d)?
      Illustrate on your graph.
      You don't need to calculate the exact quantity changes, but you should justify your answer in words using your graph.
    \vspace{3cm}

  \end{enumerate}

\end{question}
