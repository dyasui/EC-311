\begin{question}[type=exam]{4}
  \setlist{nolistsep}
  Choose the MRS for the utility function $U(x,y) = 3x + 2y + 10$:
  \begin{enumerate}[label=\alph*), noitemsep]
    \item \vary
      {$\frac{10}{3}$}
      {$\frac{3}{2}$} % correct
    \item \vary
      {$\frac{2}{10}$}
      {$\frac{10}{3}$}
    \item \vary
      {$\frac{3}{2} + 10$}
      {$\frac{2}{10}$}
    \item \vary
      {$\frac{3}{2}$} % correct
      {$\frac{3}{2} + 10$}
  \end{enumerate}
\end{question}

\begin{question}[type=exam]{4}
  \setlist{nolistsep}
  For the utility function $U(x,y) = x^{1/2}y^{1/2}$, 
  which of the following bundles would place you on the same indifference curve as the bundle $(x=4,y=4)$?
  \begin{enumerate}[label=\alph*), noitemsep]
    \item \vary
    {$(1,16)$} % correct
    {$(4,12)$}
    \item \vary
    {$(8,8)$}
    {$(1,16)$} % correct
    \item \vary
    {$(1/2,4)$}
    {$(8,8)$}
    \item \vary
    {$(4,12)$}
    {$(1/2,4)$}
  \end{enumerate}
\end{question}

\begin{question}[type=exam]{4}
  \setlist{nolistsep}
  The units of the MRS are in \rule{1cm}{0.15mm}.
  \begin{enumerate}[label=\alph*), noitemsep]
    \item \vary
      {Dollars}
      {Quantity of good $y$ per unit of good $x$} % correct
    \item \vary
      {Utils}
      {Dollars per quantity of $x$}
    \item \vary
      {Quantity of good $y$ per unit of good $x$} % correct
      {Dollars}
    \item \vary
      {Dollars per quantity of $x$}
      {Utils}
  \end{enumerate}
\end{question}

\begin{question}[type=exam]{4}
  \setlist{nolistsep}
  If $x$ is a \textbf{good} and $y$ is a \textbf{bad}, then on the $(x,y)$ plane, moving in which direction represents moving to \textbf{more preferred} bundles?
  \begin{enumerate}[label=\alph*), noitemsep]
    \item \vary
      {Down, Left}
      {Up, Right}
    \item \vary
      {Down, Right} % correct
      {Up, Left}
    \item \vary
      {Up, Left}
      {Down, Right} % correct
    \item \vary
      {Up, Right}
      {Down, Left}
  \end{enumerate}
\end{question}


\begin{question}[type=exam]{4}
  \setlist{nolistsep}
  Assuming we are not at a corner solution and the consumer is already using all of their budget,
  if $\frac{MU_x}{p_x} > \frac{MU_y}{p_y}$, what does this tell us?
  \begin{enumerate}[label=\alph*), noitemsep]
    \item \vary
      {They should consume more of $x$ and less of $y$} % correct
      {They should consume more of $y$ and less of $x$}
    \item \vary
      {They should consume more of $y$ and less of $x$}
      {They should consume more of $x$ and less of $y$} % correct
    \item \vary
      {They are maximizing their utility}
      {They are maximizing their utility}
    \item \vary
      {They should consume more of both $x$ and $y$}
      {They should consume more of both $x$ and $y$}
  \end{enumerate}
\end{question}

\begin{question}[type=exam]{4}
  \setlist{nolistsep}
  The price elasticity of demand, $\epsilon_{x^*,p_x}$, should always be:
  \begin{enumerate}[label=\alph*), noitemsep]
    \item \vary
      {Less than one}
      {Positive}
    \item \vary
      {Positive}
      {Less than one}
    \item \vary
      {Negative} % correct
      {Greater than one}
    \item \vary
      {Greater than one}
      {Negative} % correct
  \end{enumerate}
\end{question}

\begin{question}[type=exam]{4}
  \setlist{nolistsep}
  The sign of the slope of the Engel Curve (positive or negative)...
  \begin{enumerate}[label=\alph*), noitemsep]
    \item \vary
      {will always be negative}
      {will always be negative}
    \item \vary
      {depends on the cross-price elasticity of the good}
      {will always be zero}
    \item \vary
      {will always be zero}
      {depends on whether the good is normal or inferior} % correct
    \item \vary
      {depends on whether the good is normal or inferior} % correct
      {depends on the cross-price elasticity of the good}
  \end{enumerate}
\end{question}

\begin{question}[type=exam]{4}
  \setlist{nolistsep}
  Which of the following would cause the demand curve to shift \textbf{outward}?
  \begin{enumerate}[label=\alph*), noitemsep]
    \item \vary
      {Increase in consumer incomes} % correct
      {Increase in the supply of the good}
    \item \vary
      {Decrease in the price of a substitute good}
      {Increase in the supply of the good}
    \item \vary
      {Decrease in the good's own price}
    \item \vary
      {Increase in consumer incomes} % correct
      {Decrease in the good's own price}
  \end{enumerate}
\end{question}

\begin{question}[type=exam]{4}
  \setlist{nolistsep}
  Suppose a consumer is indifferent between the bundles $(x=3,y=1)$, and $(x=1,y=3)$.
  If their preferences satisfy (strict) \textit{convexity} and \textit{more is better}, 
  what can we say about their preference for the bundle $(x=2,y=2)$?
  \begin{enumerate}[label=\alph*), noitemsep]
    \item \vary
      {$(1,3)$ is preferred to $(2,2)$}
      {$(2,2)$ is preferred to $(3,1)$} % correct
    \item \vary
      {$(2,2)$ is preferred to $(3,1)$} % correct
      {$(1,3)$ is preferred to $(2,2)$}
    \item \vary
      {They are indifferent between $(2,2)$ and $(3,1)$}
      {There is not enough information to say anything about $(2,2)$}
    \item \vary
      {There is not enough information to say anything about $(2,2)$}
      {They are indifferent between $(2,2)$ and $(3,1)$}
  \end{enumerate}
\end{question}

\begin{question}[type=exam]{4}
  \setlist{nolistsep}
  A decrease in the price of good $x$ will \rule{1cm}{0.15mm} the maximum possible consumption of $x$
  and \rule{1cm}{0.15mm} the maximum possible consumption of $y$.
  \begin{enumerate}[label=\alph*), noitemsep]
    \item \vary
      {increase : leave unchanged} % correct
      {leave unchanged : increase}
    \item \vary
      {decrease : leave unchanged}
      {leave unchanged : decrease}
    \item \vary
      {leave unchanged : increase}
      {increase : leave unchanged} % correct
    \item \vary
      {leave unchanged : decrease}
      {decrease : leave unchanged}
  \end{enumerate}
\end{question}

\PrintSolutionsTF{
  \fbox{
    \parbox{\linewidth}{
      \underline{Answers:}
      \vary{
        % version 1:
        \textbf{1:} d,
        \textbf{2:} a,
        \textbf{3:} c,
        \textbf{4:} b,
        \textbf{5:} a,
        \textbf{6:} c,
        \textbf{7:} d,
        \textbf{8:} a,
        \textbf{9:} b,
        \textbf{10} a,
      }{
        % version 2:
        \textbf{1:} a,
        \textbf{2:} b,
        \textbf{3:} a,
        \textbf{4:} c,
        \textbf{5:} b,
        \textbf{6:} d,
        \textbf{7:} c,
        \textbf{8:} d,
        \textbf{9:} a,
        \textbf{10} c,
      }
    }}
  }{}
