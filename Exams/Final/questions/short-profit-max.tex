\begin{question}[type=exam]{10}
  Imagine a firm with the cost function 
  $C(Q) = \frac{1}{3} Q^3 - \vary{270}{380}Q + \vary{55}{66}$. 
  Currently, the firm can sell its product for a price of \$\vary{30}{20}.
  What is the short-run profit maximizing quantity? 
\end{question}

\PrintSolutionsTF{
  \fbox{
    \parbox{\linewidth}{
      Price taking firm's profit is maximized when $P=MR=MC$:
      \begin{align*}
        \vary{30}{20} & = \frac{3}{3} Q^{3-1} - \vary{270}{380} \\
        \vary{30}{20} + \vary{270}{380} & = Q^2 \\
        Q^2 & = \vary{300}{400} \\
        Q^S & = \sqrt{\vary{300}{400}} \vary{\approx \underline{17.32}}{=\underline{20}}
      \end{align*}
      }}
  }{\vspace{8cm}}