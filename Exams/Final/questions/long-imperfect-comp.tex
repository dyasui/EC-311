\begin{question}[type=exam]{30}
  In this question, we will look at the different outcomes when two firms compete against each other.
  There are firm 1 and firm 2 and they have identical marginal cost functions of
  $MC_1(q)=MC_2(q)=\vary{4}{3}q + \vary{16}{9}$.
  Assume that the goods produced by each firm are identical and they both face the demand curve
  $P=\vary{136}{59} - Q^D$.
  The total market quantity supplied is equal to the sum of firm 1 and firm 2's production:
  $Q^S = q_1+q_2$

  \begin{enumerate}[label=(\alph*)]

    \item (\points{8})
    First, solve for what would happen in the market if both firms act as \textbf{price-takers}.
    Solve for the total market supply curve under this assumption and use it to find the $P^*$ and $Q^*$ in this 'perfect competition' equilibrium.

    \PrintSolutionsTF{
      \fbox{\parbox{\linewidth}{

      If neither firm can affect the price, they take it as a constant in their marginal revenue:
      \begin{align*}
        P &= \vary{4}{3}q + \vary{16}{9} \\
        \Rightarrow \frac{P - \vary{16}{9}}{\vary{4}{3}} & = q
      \end{align*}
      The total quantity supplied depends on the price and can be used to find the supply curve:
      \begin{align*}
        Q^S & = \frac{P - \vary{16}{9}}{\vary{4}{3}} +  \frac{P - \vary{16}{9}}{\vary{4}{3}} \\
        Q^S + \vary{8}{6} & = \frac{2P}{\vary{4}{3}} \\
        P & = \vary{2}{1.5} Q^S + \vary{16}{9}
      \end{align*}
      So the equilibrium quantity can be found at the intersection of the supply and demand curves:
      \begin{align*}
        \vary{136}{59} - Q^D & = \vary{2}{1.5} Q^S + \vary{16}{9} \\
        \vary{136}{59} - \vary{16}{9} & = \vary{3}{2.5}Q \\
        Q_{PC}^* & = \underline{\vary{40}{20}} \\
        P_{PC}^* & = \underline{\vary{96}{39}}
      \end{align*}
      }
      }
    }{
      \vspace{7cm}
    }
    
    \item (\points{4})
    How likely do you think that the predicted market price and quantity will occur if these are the only firms
    in this market and there are high fixed costs which prevent any other firms entering the market.

    Justify your answer based on what we have discussed in the second half of this class.

    \PrintSolutionsTF{
      \fbox{\parbox{\linewidth}{
        Answers may vary but should include some discussion of one of the models of imperfect competition from class.

        For example, the price-taking equilibrium may be more likely if firms have to bid on prices (Bertrand model).
        In that model, if neither firm has an advantage in marginal costs, then they could both arrive at a Nash equilibrium in which they bid exactly their marginal cost
        because they know that any bid above that can be undercut by their rival.

        Alternatively, the price-taking equilibrium may be unlikely if firms can bid on quantities like in Cournot or Stackelberg models.

        These firms may be able to collude and jointly set their quantities as if they were a monopoly, 
        although this depends on how much they trust each other and how much they care about sustaining the cooperative outcome in the future.
      }
      }
    }{}
    \newpage

    Now consider a Cournot duopoly model with the same two firms competing based on quantities.

    \item (\points{3})
    Write out the profit function for firm 1 based on their own production strategy $q_1$ as well as their rival firm's quantity $q_2$.
    Assume that they have to choose $q_1$ at the same time that firm 2 sets $q_2$.

    \PrintSolutionsTF{
      \fbox{\parbox{\linewidth}{

      \begin{align*}
        \pi_1(q_1,q_2)  & = (\vary{136}{59} - q_1 - q_2) q_1 - (\frac{\vary{4}{3}}{2}q_1^2 + \vary{16}{9}q_1) \\
      \end{align*}

      \hrulefill

      {\color{red} 
      I made this prompt more confusing than I meant to. If they used the Marginal Cost instead of the Total cost function in the profit here, give them credit if they get the answer in red.
      \begin{align*}
        \pi_1(q_1,q_2)  & = (\vary{136}{59} - q_1 - q_2) q_1 - (\vary{4}{3}q_1 + \vary{16}{9}) \\
      \end{align*}
      }


      \textbf{Grading Note:} Anything of the form $\pi = P\cdot q - C(q)$ is correct 
      as long as the student states in math or in writing that the total market $Q$ is the sum of both firms' individual $q$
      (alternatively that the market price depends on both firm's quantity).
      }
      }
    }{\vspace{4cm}}

    \item (\points{4})
    Use your answer from part (c) to derive a best-response rule 
    that tells firm 1 how much $q_1$ they should produce as a function of $q_2$.

    \PrintSolutionsTF{
      \fbox{\parbox{\linewidth}{
      \begin{align*}
        \frac{\delta \pi}{\delta q_1} = 0  \Rightarrow \vary{136}{59} - 2q_1 - q_2 & = \vary{4}{3}q_1  + \vary{16}{9} \\
        \Rightarrow \vary{6}{5}q_1  & = \vary{120}{50} - q_2 \\
        BR_1(q_2) & = \vary{\frac{120-q_2}{6}}{\frac{50-q_2}{5}} 
      \end{align*}

      \hrulefill

      {\color{red}
      \begin{align*}
        q_2 & = \frac{\vary{132}{56}-q_1}{2} 
      \end{align*}
      }

      \textbf{Grading Note:} Give partial credit if the setup of taking the partial derivative of the profit from part (a) with respect to $q_1$ is there.
      Some students will probably get lost in the algebra of isolating $q_1$ on one side of the equation, but try to give them most of the points 
      if they explain what they are trying to solve for (which is $q_1$ as a function of $q_2$).
      }
      }
    }{\vspace{10cm}}

  \item (\points{3})
  What is firm 2's best response rule?

    \PrintSolutionsTF{
      \fbox{\parbox{\linewidth}{
      You could go through the same process as part (d), 
      or you could just say that because the firms have the same marginal cost functions and revenue structure,
      that the game is symmetric.
      
      So,
       $$BR_2(q_1) = \vary{\frac{120-q_1}{6}}{\frac{50-q_1}{5}}$$

      {\color{red}
      \begin{align*}
        q_2 & = \frac{\vary{132}{56}-q_1}{2} 
      \end{align*}
      }
      }
      }
    }{\vspace{3cm}}

    \newpage

  \item (\points{8})
  Use your best response functions to solve for a Nash equilibrium in the simultaneous Cournot competition game.

    \PrintSolutionsTF{
      \fbox{\parbox{\linewidth}{
      A Nash equilibrium is where the best response rules intersect:
      \begin{align*}
        q_1  & = \vary{\frac{120-(\frac{120-q_1}{6})}{6}}{\frac{50-\frac{50-q_1}{5}}{5}} \\
        \vary{6}{5}q_1 & = \vary{120}{50} - (\frac{\vary{120}{50}-q_1}{\vary{6}{5}}) \\
        \frac{\vary{36}{25}q_1 - q_1}{\vary{6}{5}} & = \vary{120}{50} - \vary{20}{10} \\
        \vary{35}{24}q_1 & = \vary{600}{200} \\
        q_1^* & = \vary{\frac{600}{35}}{\frac{200}{24}} 
      \end{align*}
      Then plug in $q_1^* = \vary{\frac{600}{35}}{\frac{200}{24}}$ into $BR_2(q_1^*)$:
      \begin{align*}
        q_2^* & = BR_2(q_1) = \vary{\frac{120-\frac{600}{35}}{6}}{\frac{50-\frac{200}{24}}{5}} \\
      \end{align*}
      So the Nash equilibrium of this game is:
      $$\{ q_1^* = \vary{600/35}{200/24}, ~ q_2^* = \vary{600/35}{200/24} \}$$

      As a decimal, $\vary{600/35}{200/24} \approx \vary{17.1}{8.3}$

      \hrulefill

      {\color{red}
      $$\{ q_1^* = \vary{44}{56/3}, ~ q_2^* = \vary{44}{56/3} \}$$
      }

      \textbf{Grading Note:} Final answer can be expressed in fractions or rounded decimal.
      Give full credit if the quantities are within one or two tenths of the correct answer.
      There are lots of ways to get lost in the algebra here,
      but the student should be able to use what they know about Cournot from class 
      to see that the total of $q_1+q_2$ should be less than the $Q^*_PC$ they found in part (a).
      Give most of the points if it seems like the student understands the set-up and intuition of the model
      and only algebra mistakes held them back from finding the correct answer.
    
      }
      }
    }{\vspace{3cm}}
  \end{enumerate}
\end{question}
