% Assesment: Production (Easy)
\begin{question}[type=exam]{3}
  What type of returns to scale does the production function
  $F(L,K)=L^{2/5}K^{1/5}$ feature?
  \begin{enumerate}[label=\alph*), noitemsep]
    \item \vary
      {increasing}
      {not enough information}
    \item \vary
      {constant}
      {increasing}
    \item \vary
      {decreasing} % correct
      {constant}
    \item \vary
      {not enough information}
      {decreasing} % correct
  \end{enumerate}
\end{question}

% Assesment: Production (Hard)
\begin{question}[type=exam]{3}
  Suppose a firm with the production function
  $F(L,K)=L^{2}K^{3}$ 
  is currently using 10 machines and 10 workers ($K=L=10$).
  If the amount of labor decreases by 1 worker,
  how much \textbf{Capital} (approximately) would have to be added
  to keep the original level of production?
  \begin{enumerate}[label=\alph*), noitemsep]
    \item \vary
      {2 machines}
      {1/2 machine}
    \item \vary
      {1/2 machine}
      {2 machines}
    \item \vary
      {3 machines}
      {2/3 machine} % correct
    \item \vary
      {2/3 machine} % correct
      {3 machines}
  \end{enumerate}
\end{question}

% Assement: Costs (Easy)
\begin{question}[type=exam]{3}
  Currently, a firm with a cost function of $C(q)=\frac{1}{3}q^3$
  is producing 10 units in a perfectly competitive market
  in which the market price is \$10 per unit.
  Is the firm producing \textit{too much}, \textit{too little},
  or are they \textit{profit-maximizing?}
  \begin{enumerate}[label=\alph*), noitemsep]
    \item \vary
      {Too much} % correct
      {Too little}
    \item \vary
      {Not enough information}
      {Profit-maximizing} 
    \item \vary
      {Profit-maximizing}
      {Not enough information}
    \item \vary
      {Too little}
      {Too much} % correct
  \end{enumerate}
\end{question}

% Assement: Costs (Hard)
\begin{question}[type=exam]{3}
  What is the \textbf{average total cost function} for a firm with the cost function
  $C(q)=3q^3+6q^2-10q+18$?
  \begin{enumerate}[label=\alph*), noitemsep]
    \item \vary
      {$3q^3+6q^2-10q$}
      {$9q^3+12q^2-10$}
    \item \vary
      {$9q^3+12q^2-10$}
      {$18/q$}
    \item \vary
      {$3q^2+6q^1-10+18/q$} % correct
      {$3q^3+6q^2-10q$}
    \item \vary
      {$18/q$}
      {$3q^2+6q^1-10+18/q$} %correct
  \end{enumerate}
\end{question}

% Assement: Imperfect Competition (Easy)
\begin{question}[type=exam]{3}
  How do monopolistic markets compare to perfectly competitive markets?
  \begin{enumerate}[label=\alph*), noitemsep]
    \item \vary
      {monopolies lead to higher price and higher quantity}
      {monopolies lead to lower price and lower quantity}
    \item \vary
      {monopolies lead to higher price and lower quantity} % correct
      {monopolies lead to higher price and higher quantity}
    \item \vary
      {monopolies lead to lower price and lower quantity}
      {monopolies lead to higher price and lower quantity} % correct
    \item \vary
      {monopolies lead to lower price and higher quantity}
      {monopolies lead to lower price and lower quantity}
  \end{enumerate}
\end{question}

\PrintSolutionsTF{
  \fbox{
    \parbox{\linewidth}{
      \underline{Answers:}
      \vary{
        % version 1:
        1: \textbf{c},
        2: \textbf{d},
        3: \textbf{a},
        4: \textbf{c},
        5: \textbf{b}
      }{
        % version 2:
        1: \textbf{d},
        2: \textbf{c},
        3: \textbf{d},
        4: \textbf{d},
        5: \textbf{c}
      }
    }}
  }{}

\newpage

% Assement: Imperfect Competition (Hard)
\begin{question}[type=exam]{3}
  For a \textbf{monopolist} with a marginal cost function of $C(Q)=9Q^2$
  who faces a market demand curve of $P=20-Q$,
  solve for their profit-maximizing quantity.
  \begin{enumerate}[label=\alph*), noitemsep]
    \item \vary
      {1/2}
      {20}
    \item \vary
      {20}
      {1} % correct
    \item \vary
      {2} 
      {2} 
    \item \vary
      {1} % correct
      {1/2}
  \end{enumerate}
\end{question}

\begin{question}[type=exam]{3}
  Imagine a short-run market made up of 10 firms
  who each have an individual supply curve of $q_s=1/5P$.
  If the demand curve is $Q_D=100-8P$,
  what is the market equilibrium price?
  \begin{enumerate}[label=\alph*), noitemsep]
    \item \vary
      {$P^*=8$}
      {$P^*=10$} % correct
    \item \vary
      {$P^*=1/5$}
      {$P^*=2$}
    \item \vary
      {$P^*=2$}
      {$P^*=1/5$}
    \item \vary
      {$P^*=10$} % correct
      {$P^*=8$}
  \end{enumerate}
\end{question}

\begin{question}[type=exam]{3}
  Which of the following would cause the short-run supply curve to shift \textbf{outward}?
  \begin{enumerate}[label=\alph*), noitemsep]
    \item \vary
      {Decrease in input prices} % correct
      {Decrease in number of firms}
    \item \vary
      {Increase in input prices}
      {Decrease in input prices} % correct
    \item \vary
      {Decrease in number of firms}
      {Increase in consumer incomes}
    \item \vary
      {Increase in consumer incomes}
      {Increase in input prices}
  \end{enumerate}
\end{question}

\begin{question}[type=exam]{3}
  If firms are producing \textit{above} the minimum of their \textbf{average total cost} curve,
  what will happen in the \textbf{long-run}?
  \begin{enumerate}[label=\alph*), noitemsep]
    \item \vary
      {Existing firms will \textit{exit} the market and drive the price \textbf{up}}
      {New firms will \textit{enter} the market and drive the price \textbf{up}}
    \item \vary
      {New firms will \textit{enter} the market and drive the price \textbf{up}}
      {Existing firms will \textit{exit} the market and drive the price \textbf{down}}
    \item \vary
      {New firms will \textit{enter} the market and drive the price \textbf{down}} % correct
      {Existing firms will \textit{exit} the market and drive the price \textbf{up}}
    \item \vary
      {Existing firms will \textit{exit} the market and drive the price \textbf{down}}
      {New firms will \textit{enter} the market and drive the price \textbf{down}} % correct
  \end{enumerate}
\end{question}

\begin{question}[type=exam]{3}
  Which type of competition has firms compete on \textit{price-setting}
  instead of \textit{quantity-setting}?
  \begin{enumerate}[label=\alph*), noitemsep]
    \item \vary
      {Cournot}
      {Monopoly}
    \item \vary
      {Monopoly}
      {Bertrand} % correct
    \item \vary
      {Bertrand} % correct
      {Stackelberg}
    \item \vary
      {Stackelberg}
      {Cournot}
  \end{enumerate}
\end{question}

\PrintSolutionsTF{
  \fbox{
    \parbox{\linewidth}{
      \underline{Answers:}
      \vary{
        % version 1:
        6: \textbf{d},
        7: \textbf{d},
        8: \textbf{a},
        9: \textbf{c},
        10: \textbf{c}
      }{
        % version 2:
        6: \textbf{b},
        7: \textbf{a},
        8: \textbf{b},
        9: \textbf{d},
        10: \textbf{b}
      }
    }}
  }{}
