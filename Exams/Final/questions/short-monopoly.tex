\begin{question}[type=exam]{10}
  Consider the behavior of a monopolist firm with no competitors.
  Suppose the demand curve they face is $P=\vary{550}{720} - Q$ where the single firm sets the entire market quantity supplied, $Q$.
  Also, for simplicity suppose that the firm's marginal cost is zero dollars per unit produced.

  \begin{itemize}
    \item Derive the monopolist firm's \textbf{marginal revenue} function.
    \item Use this to \textbf{solve for maximum amount of profit} (in dollars) they could earn.
  \end{itemize}
\end{question}

\PrintSolutionsTF{
  \fbox{\parbox{\linewidth}{
    Marginal revenue comes from the derivative of total revenue:
    \begin{align*}
     TR & = (\vary{550}{720} - Q) \cdot Q \\
     MR & = \vary{550}{720} - 2Q \\
    \end{align*}
    Monopolist profit maximized when $MR = MC$:
    \begin{align*}
      \vary{550}{720} - 2Q & = 0 \\
      \vary{550}{720} & = 2Q \\
    Q^S_M = \vary{550}{720} / 2 & = \underline{\vary{275}{360}} \\
    \end{align*} 
    Use the monpolist's optimal quantity to solve for their max profit:
    \begin{align*}
      \pi_M^* & = (\vary{550}{720} - (\vary{275}{360}))\cdot \vary{275}{360} \\
      & = \underline{\vary{75625}{129600}}
    \end{align*}
  }}
}{
  \vspace{5cm}
}