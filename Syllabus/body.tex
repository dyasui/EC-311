\section{Course Description}

The objective of this course is to provide you with an introduction to, and overview of, the
most important concepts in microeconomics. The material in this course provides the foundation
for studying issues in a wide variety of fields in economics, including international trade, labor,
development, and other fields, and is essential for further study in economics.

We start with a review of material from Introduction to Microeconomics (EC 201) by discussing
the supply and demand model. We proceed to study consumer theory, gaining an understanding
of how consumers’ preferences and budgets lead to their consumption choices and the derivation
of demand curves. Having gained an understanding of both the demand and the supply sides of
economies, we proceed by studying the interactions of consumers and firms in different types of
markets.

\subsection*{Learning Objectives:}

\begin{itemize}
\item Proficiency in microeconomic analysis.
  This builds on the basic microeconomic concepts and requires the ability to examine models of agents’ decisions, including consumer utility maximization, firm profit maximization, and market equilibria, using mathematical tools.
  \item Develop the ability to perform constrained optimization in both the economic and critical thinking skills.
  \item Upon completing the course students should feel comfortable solving mathematical problems that allow them to build basic models of markets and using their intuitive understanding to explain the relationship between market inputs like preferences, technologies, costs and market outputs like price and quantity.
\end{itemize}

\subsection*{Prerequisites:}

EC 201 (Intro to Microeconomics) and MATH 111 (College Algebra) or equivalent.
This course makes extensive use of mathematics.
Given the nature of the material, we will become comfortable with solving equations, taking derivatives, and maximizing a function.
We will go through a crash course in the necessary derivative skills in the first lecture.

\subsection*{Required Materials:}

\textbf{\textit{Microeconomics, $4^{th}$ Edition} by Goolsbee, Levitt, and Syverson}.

The book comes with access to “Achieve (Macmillan Achieve)”, an online platform with practice problems, homework, e-book chapters, learning modules, and extra resources.
The course code is xu5rmk.
It is available at the Duck Store (ISBN: 9781319604141)

\section{Assignments and Grading} 

\begin{center}\begin{minipage}{3.8in}\begin{flushleft}
    \hyperlink{grading_participation}{Participation} \dotfill 5\% \\
    \hyperlink{grading_HW}{Homework}         \dotfill 20\% \\
    \hyperlink{grading_quiz}{Canvas Quizzes} \dotfill 15\% \\
    \hyperlink{grading_exam}{Midterm Exam}   \dotfill 30\% \\
    \hyperlink{grading_exam}{Final Exam}     \dotfill 30\% \\
\end{flushleft}\end{minipage}\end{center}

\hypertarget{grading_exam}{\subsection{Exams}}
Exams are closed book and may consist of a mix of multiple choice, short answer, and long answer questions.
Only non-programmable calculators are allowed during exams.
This means no graphing calculators.
I will provide as many department calculators as I can but cannot promise there will be enough for all.
Please bring your student ID to all exams.
I will award partial credit on short and long answer questions as I see fit, so always attempt them and you are encouraged to show your work.
The final is technically cumulative.
This primarily means that concepts learned during the second half of the course will expand on those from the first half.
Some older material will be fundamental to properly mastering the newer material.
All accommodations documented through the AEC will be honored.

\subsection*{No Makeup Exams}
If you know that you will miss an exam you should tell me as soon as possible.
Rather than create a new exam, my standard practice is to put the weight of the mixed exam on subsequent exams
(i.e. You miss the midterm so the final exam will now be worth 60\% of your grade).
Unless there are some unforeseen circumstances during the term, the exams will be on dates specified within this document.

\hypertarget{grading_HW}{\subsection{Homework}}
There will be a total of ~8 Homework Assignments with over 10 questions each.
Homework will be assigned via the Achieve online platform.
Assignments will be \textbf{due at 11:59pm Monday} of the week after they are assigned, except for the final assignments will be due by 11:59 pm on the Saturday before finals week.
You will have two attempts per question, with a 5\% penalty on the second attempt.
Achieve is set up to allow for homework to be turned in up to 1 day late, with a 5\% penalty.
You are able to submit assignments early and I encourage you to not wait until the last moment, given that questions may arise.
I will drop your lowest score.
This system means that there will be no possibility for late submissions (beyond the grace period) or makeups.
 
\hypertarget{grading_quiz}{\subsection{Quizzes}}
There are 7 Canvas quizzes which are designed to prepare you for the type of questions you will see on the exams.
It will consist of up to 10 questions, sometimes all from the same chapter, sometimes spread across chapters.
You will have 1 attempt for each quiz. I will drop your lowest score.
This system means that there will be no possibility for late submissions or makeups.

\hypertarget{grading_participation}{\subsection{In-class Participation}}
Lectures are setup to be interactive with students.
I will regularly ask questions throughout the class.
You will have time to attempt them and discuss most questions with classmates.
I will then randomly call on students to share their answer.
Being correct is not important, attempting it is what matters.
The purpose of these questions is three-fold:
First, because this class is largely mathematical any practice with active feedback is beneficial to the learning process.
Second, it gives me an understanding of how well the class is mastering the material.
Third, it gives you an additional incentive to come to and actively be engaged in class.
